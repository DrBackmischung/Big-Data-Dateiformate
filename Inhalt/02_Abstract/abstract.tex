\renewcommand{\abstractname}{Abstract} % Veränderter Name für das Abstract
\begin{abstract}
\begin{addmargin}[1.5cm]{1.5cm}        % Erhöhte Ränder, für Abstract Look
\thispagestyle{plain}                  % Seitenzahl auf der Abstract Seite

\begin{center}
\small\textit{Mathis Neunzig - 2240587}             % Angabe der Sprache für das Abstract
\end{center}

\begin{center}
\small\textit{- Deutsch -}             % Angabe der Sprache für das Abstract
\end{center}

\vspace{0.25cm}
Das Thema Big Data wird in der globalisierten und modernen Welt immer wichtiger. Dadurch wachsen global die von Unternehmen, Privatpersonen und Organisationen gesammelten Daten auf unvorstellbare Größen an, die es mit Software zu verarbeiten gilt. Für die Verarbeitung solcher Daten sind spezielle Dateiformate erforderlich, die auf solche Zwecke ausgerichtet sind. 

\vspace{0.25cm}
Die Arbeit beschäftigt sich mit der Frage, in welchen Punkten sich gängige Big-Data-Dateiformate voneinander unterscheiden und wann welche eingesetzt werden sollen. Dabei werden verschiedene Dateiformate miteinander verglichen. Diese Dateiformate sind entweder in der Literatur als ''Big-Data-Dateiformate'' gekennzeichnet oder sind im Alltag im Einsatz und könnten für große Daten theoretisch benutzt werden. 

\vspace{0.25cm}
Die Dateiformate Parquet, Avro und ORC heben sich deutlich von Formaten wie CSV, JSON und XML ab, da diese kleiner, schneller und performanter sind. Parquet, Avro und ORC unterscheiden sich untereinander in einigen Punkten, weshalb für den konkreten Einsatz ein Dateiformat nach Analyse und Vergleichen ausgewählt werden sollte. 

\end{addmargin}
\end{abstract}