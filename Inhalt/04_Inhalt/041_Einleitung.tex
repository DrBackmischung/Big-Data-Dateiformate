\chapter{Einleitung}
Diese Seminararbeit ist im Zuge des Moduls ''Integrationsseminar'' entstanden, welches sich mit ausgewählten Aspekten der Wirtschaftsinformatik beschäftigt. Diese Arbeit speziell vergleicht wichtige Big-Data-Dateiformate miteinander.

Mit dem Voranschreiten der digitalen Transformation und der Globalisierung wird das Thema Big Data immer wichtiger. Dadurch, dass in den verschiedensten Bereichen der Wirtschaft immer mehr Daten gebraucht werden, müssen diese Daten effizient und schnell verarbeitet werden \cite{verbraucherzentralede_geschichte_nodate}. Dabei steht ''Big Data'' für die große Datenmenge, die menschlich nicht erfassbar ist und moderne Entscheidungsmechanismen übernimmt \cite{sap_was_nodate}. 

Zu Beginn werden die Dateiformate vorgestellt, bevor Aufgrund von bestimmten Kriterien eine Auswahl getroffen wird, welche Dateiformate als Big-Data-Dateiformate klassifiziert werden könnten und somit verglichen werden. Diese werden im Anschluss miteinander im Bezug auf verschiedene Charakteristiken, die im weiteren Verlauf erläutert werden, untersucht und verglichen. Die Daten und Fakten, auf denen die Vergleiche basieren, stammen aus einer breit gefächerten Literaturrecherche zu den einzelnen Dateiformaten, sowie Verbindungen und Vergleiche zwischen diesen. 
