\chapter{Theoretische Grundlagen}

\section{Apache Parquet}
Parquet ist ein nicht von Menschen lesbares, codiertes Datei-Format für die Speicherung von Big Data und ist in der Lage, große Mengen an Daten zu speichern und diese wieder schnell zu lesen \cite{apache_apache_nodate} \cite[S. 906]{gohil_compendious_2022}. Es ordnet die Daten in Spalten, was die Lesezeit erheblich reduzieren und die Performance verbessern kann\footnote{Siehe Abbildung \ref{fig:Parquet}} \cite[S. 339]{munir_cost-based_2020}. Ein exemplarischer Aufbau ist in Anhang in der Abbildung \ref{fig:Parquet} zu finden. Es unterstützt auch fortgeschrittene Funktionen wie Prädikat-Pushdown, bei dem nur die relevanten Daten gelesen werden und Datenkomprimierung, um die Größe der Daten zu minimieren \cite[S. 268]{plase_comparison_2017}. Es verwendet ein festes Schema, was bedeutet, dass die Struktur der Daten vorab definiert ist und sich nicht mehr ändern kann. Dies hilft in der Optimierung der Performance und der Datenverwaltung \cite{apache_apache_nodate}.

\section{Apache Avro}
Apache Avro ist ein Datenserialisierungsformat, das entwickelt wurde, um Daten effizient zu serialisieren, zu deserialisieren und zu verarbeiten \cite{apache_avro_nodate}. Die Daten in Avro-Dateien sind, ähnlich wie bei CSV, reihenweise abgespeichert\footnote{Siehe Abbildung \ref{fig:Avro}}, kann jedoch durch die Codierung des Inhalts wie bei Parquet nicht von Menschen gelesen werden \cite[S. 338]{munir_cost-based_2020} \cite[S. 906]{gohil_compendious_2022}. Ein exemplarischer Aufbau einer Avro-Datei ist im Anhang in der Abbildung \ref{fig:Avro} zu finden. Avro verwendet ein festes Schema, um die Datenstruktur und -integrität sicherzustellen. Das Schema wird in der Datei selbst gespeichert, was es ermöglicht, die Daten ohne Kenntnis des Schemas zu lesen und zu verarbeiten. Avro unterstützt auch die Datenkompression, um die Datengröße zu reduzieren und die Übertragungszeit zu verkürzen \cite[S. 268]{plase_comparison_2017}. Es ist für die Verwendung mit Big-Data-Verarbeitungsrahmen wie Apache Kafka optimiert \cite[S. 906]{gohil_compendious_2022}.

\section{ORC}
Apache \ac{ORC} speichert Daten in einer spaltenorientierten Weise\footnote{Siehe Abbildung \ref{fig:ORC}}, wodurch es gezielt nur die benötigten Spalten lesen und verarbeiten kann, anstatt die gesamte Datenzeile \cite[S. 339]{munir_cost-based_2020} \cite{apache_languagemanual_nodate} \cite[S. 906]{gohil_compendious_2022}. Wie Avro und Parquet ist ORC codiert und in roher Form nicht von Menschen lesbar. Ein exemplarischer Aufbau kann in Anhang in der Abbildung \ref{fig:ORC} gefunden werden. Durch den Aufbau kann die Lesezeit erheblich reduzieren und die Performance verbessert werden. Es unterstützt Datenkomprimierung, um die Größe der Daten zu minimieren \cite[S. 268]{plase_comparison_2017}. ORC ist ein Dateiformat für den Einsatz mit Big-Data-Verarbeitungsrahmen wie Apache Hive. 

\section{Sonstige Dateiformate}
\subsubsection{XML}
\ac{XML} ist ein von Menschen lesbares, standardisiertes Format zur Beschreibung und Übertragung von Daten. Es basiert auf einer hierarchischen Struktur von Tags (Markups) und Attributen, die die Bedeutung und Struktur der Daten beschreiben\footnote{Siehe Codebeispiel \ref{code:XML}}. XML ermöglicht es Entwicklern, eigene Tags und Attribute zu definieren, um die Daten passgenau beschreiben zu können \cite[S. 7ff.]{nolan_xml_2014} \cite[S. 906]{gohil_compendious_2022}.

\subsubsection{JSON}
\ac{JSON} ist ein standardisiertes Format für die Übertragung und Speicherung von Daten, welches wie XML auch von Menschen lesbar ist. Es basiert auf der Syntax von JavaScript und verwendet eine klare, textbasierte Notation, um Daten in einer Hierarchie von Schlüssel-Wert-Paaren darzustellen\footnote{Siehe Codebeispiel \ref{code:JSON}} \cite[S. 4]{belov_experimental_2021} \cite[S. 14ff.]{nolan_xml_2014} \cite[S. 906]{gohil_compendious_2022}.

\subsubsection{CSV}
\ac{CSV} ist ein einfaches Textdatei-Format zur Speicherung von Tabellendaten. Die Daten werden in Form von Zeilen und Spalten gespeichert, wobei die Spalten durch Kommas voneinander getrennt sind\footnote{Siehe Codebeispiel \ref{code:CSV}}. Jede Zeile entspricht einer Datenzeile, wobei die erste Zeile den Namen der Spalten enthalten kann \cite[S. 905]{gohil_compendious_2022}.
