\chapter{Zusammenfassung}
Zusammenfassend wurde gezeigt, dass die wichtigen ''Big-Data-Dateiformate'' Parquet, Avro und ORC sich untereinander verschieden sind. Die Dateiformate bieten verschiedene Vor- und Nachteile, welche bei dem konkreten Einsatz berücksichtigt und nach diesen das passende Dateiformat gewählt werden soll.

Avro ist vor allem für Operationen gut geeignet, die viele Schreib-Anforderungen benötigen. ORC und Parquet sind langsamer im schreiben als Avro und dementsprechend für Lese-lastige Operationen geeignet. Die Wahl des richtigen Dateiformats hängt jedoch auch sehr stark von der gewählten Anwendung ab. Ob Spark, Impala, Kafka oder andere Programme genutzt werden, hat einen großen Einfluss auf die Kompatibilität mit den Dateiformaten, da diese für bestimmte Programme entwickelt und optimiert worden sind. 

Neben den oben beschriebenen wichtigen Vergleichen sind die übrigen Vergleiche bei allen drei Dateiformaten nicht so stark ausgefallen. Vor allem die Kompression, die Programmierung, die Nutzung eines Schemas und weitere Vergleiche sind keine Ausschlusskriterien, ein Dateiformat nicht zu nutzen. Deswegen sollte die potentielle Wahl eines Dateiformats vor allem daran entschieden werden, ob mehr Schreib- oder Lese-Operationen genutzt werden und welches Tool zum Einsatz kommt. CSV, JSON und XML soll nicht genutzt werden, solange es sich um Daten im Umfeld von Big Data handelt.